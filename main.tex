\documentclass{beamer}
\usepackage{xcolor}
\usepackage[utf8]{inputenc}
\usepackage[english]{babel} 
\usepackage{listings}

\usepackage{parcolumns}


\usetheme{Madrid}
\usecolortheme{beaver}

\beamertemplatenavigationsymbolsempty

\renewcommand{\emph}{\textcolor{red}}

%------------------------------------------------------------
%This block of code defines the information to appear in the
%Title page
\title[Resource Sharing] %optional
{Enabling resource sharing in dataflow circuits}

\author[Marmet] % (optional)
{Axel Marmet}

\institute[EPFL] % (optional)

\date[EPFL 2019] % (optional)

%\logo{\includegraphics[height=1cm]{EPFL_Logo_Digital_RGB_PROD-768x333.png}}

%End of title page configuration block
%------------------------------------------------------------



%------------------------------------------------------------
%The next block of commands puts the table of contents at the 
%beginning of each section and highlights the current section:

%\AtBeginSection[]
%{
%  \begin{frame}
%    \frametitle{Table of Contents}
%    \tableofcontents[currentsection]
%  \end{frame}
%}
%------------------------------------------------------------


\begin{document}

%The next statement creates the title page.
\frame{\titlepage}


\begin{frame}
\frametitle{Table of Contents}
\tableofcontents
\end{frame}

\section{Running example}
\begin{frame}[fragile]
\frametitle{Motivation}
\begin{columns}[T]
    \begin{column}{0.45\textwidth}
      \begin{itemize}
          \item New iteration every two clock cycles
          \item Each multiplier has an occupancy of $0.5$
          \item Using only one multiplier would not hurt performance but diminish size of circuit
      \end{itemize}
    \end{column}
    \begin{column}{0.1\textwidth}
    \end{column}
    \begin{column}{0.45\textwidth}
      \includegraphics[scale=0.28]{base_case.png}
    \end{column}
  \end{columns}
\end{frame}

\begin{frame}[fragile]
\frametitle{Initial idea}
\begin{columns}[T]
    \begin{column}{0.45\textwidth}
    We add two new components to enable sharing \newline
    \begin{description}
        \item[Selector] Responsible for selecting inputs in a way that ensures fairness and will never deadlock
        \item[Distributor] Must send the resulting token to the correct outputs, is told where to send by the Selector through a FIFO
    \end{description}
    \end{column}
    \begin{column}{0.1\textwidth}
    \end{column}
    \begin{column}{0.45\textwidth}
      \includegraphics[scale=0.28]{shared_base_case.png}
    \end{column}
  \end{columns}
\end{frame}

\section{Deadlock avoidance}
\begin{frame}{Deadlock avoidance}
  How to schedule execution without causing deadlock ?
  \begin{columns}[T]
    \begin{column}{0.25\textwidth}
      \includegraphics[scale=0.25]{blocking_unshared.png}
    \end{column}
    \begin{column}{0.1\textwidth}
    \begin{center}
        becomes
    \end{center}
    \end{column}
    \begin{column}{0.25\textwidth}
      \includegraphics[scale=0.25]{blocking_shared.png}
    \end{column}
  \end{columns}
\end{frame}

\begin{frame}{Enforcing all transactions firing}
\begin{columns}
    \begin{column}{0.45\textwidth}
    We must enforce that all transactions from one BB happen before any from another BB (or the same in another iteration) \newline \newline
    We use the same mechanism as LSQs to be informed of the order of execution of the BBs
    \end{column}
    \begin{column}{0.1\textwidth}
    \end{column}
    \begin{column}{0.45\textwidth}
      \includegraphics[scale=0.25]{blocking_shared_solution.png}
    \end{column}
  \end{columns}
\end{frame}

\begin{frame}{Ordering}
For optimal buffer placement we want to be able to somehow indicate to the MILP model that the components are shared. The first idea was to modify the latencies and initiation intervals. Not ideal because incrementing latency correctly is not trivial. The second idea is to add to the control path to explicitly enforce an order per BB. (There won't be a lengthy paragraph like that in the final presentation, this is just to show how I wanna structure it)
\end{frame}

\begin{frame}{Explicit ordering for MILP}
Nice picture here
\end{frame}

\section{Components}
\begin{frame}{Selector}
Todo
\end{frame}

\begin{frame}{Distributor}
    \includegraphics[scale=0.35]{distributor.png}
\end{frame}

\section{Algorithm}
\begin{frame}{Algorithm}
Simplified pseudocode algorithm
\end{frame}

\section{Next steps}
\begin{frame}{Next steps}
Find a heuristic or algorithm to find best ordering without exhaustive search
\end{frame}

\end{document}